\documentclass[10pt,a4paper,notitlepage]{scrreprt}
\usepackage[utf8]{inputenc}
\usepackage{amsmath}
\usepackage{amsfonts}
\usepackage{amssymb}
\usepackage{graphicx}
\usepackage{xcolor}
\usepackage{geometry}
\usepackage{ngerman}
\usepackage{textcomp}
\usepackage[autostyle=true,german=quotes]{csquotes}
\usepackage{multirow}
\usepackage{setspace}

\begin{document}
	\newgeometry{bottom=3cm}
	
	\pagestyle{headings}
	\onehalfspacing
\centering 	{\Huge Konzeptentwurf}\\
		{\large Komplexpraktikum Medieninformatik - Multimediatechnologie\\}
\
\\
\centering 	05.02.2016\\\
\\
\centering 	Gruppe 2\\\
		Gruppenleiter: Alexandra Krien\\
		Georg Eckert\\
		Stephanie Sara Groß\\
		Philipp Roscher\\\

		Ehemals: Nathalie Blasberg

\tableofcontents

\begin{flushleft}
\chapter{Einleitung}

Wir sind Gruppe 2, die im Rahmen des Komplexpraktikums Medieninformatik, Teilgebiet Multimediatechnologie, eine Spielsoftware entwickelt. 
Unsere Gruppenmitglieder sind dem Titelblatt zu entnehmen.\

Aufgabe des Komplexpraktikums ist es, in einer Gruppe von fünf Studenten eine Software zu entwickeln.
Spezifisch soll es sich dabei um ein Multiplayer-Spiel handeln, welches das Prinzip der 'Game Orchestration' einbezieht.\\
Bei diesem Konzept handelt es sich um eine Form der Rollenverteilung innerhalb eines Spiels.
Dabei übernimmt einer der Spieler die Rolle des Game Masters der in besonderer Art und Weise Einfluss auf das Spielgeschehen nehmen kann.\\
Dies kann verschiedene Auswirkungen auf die Position des Game Masters gegenüber den weiteren Spielern haben. So kann er sowohl
unterstützend, als auch als Antagonist auftreten.\\
In unserem Spielkonzept entschieden wir uns für die Rolle des Antagonisten.\\
Weitere Vorgaben beschränken sich lediglich auf das Framework libGDX, sowie die Nutzung einer Netzwerkkommunikation. Genauer gehen wir darauf im Bereich 'Technologie' ein.

\chapter{Konzept}

\section{Story}
In einem Labyrinth tief unter der Erde, lebt ein Drache und beschützt dort seit Jahren einen Schatz. Kämpfer aus dem ganzen Land reisen an um sich den Schatz unter den Nagel zu reißen. Erst wenn sie den Drachen und andere Widersacher besiegt und alle Schlüsselteile gesammelt haben, können sie den Schatz ergattern und aus dem Labyrinth entkommen.\\

\section{Setting}
Es nehmen 5 Spieler teil. Dafür werden zwei Gruppen und ein Game Master durch Zufall entschieden.\\
Der Game Master ist der Drache und muss den Schatz vor den anderen zwei Gruppen beschützen. Dafür hat er verschiedene Fähigkeiten: Als Herrscher des Labyrinths hat er Überblick über alle Kammern, kann geheime Pfade nutzen und Monster in dem Kampf schicken. Da er viel strategischer handelt als die anderen Spieler und weniger Lebenspunkte hat, legt er mehr Wert auf seine Verteidigung.\\
Die zwei Teams bestehen aus jeweils zwei Spielern. Nachdem die Aufteilung durch Zufall entschieden wurde, können die Teilnehmer ihre Spielfigur auswählen. Je nach Klasse liegen unterschiedliche Angriffe vor.\\
Nach der Auswahl fängt das Spiel im Labyrinth an. \\
Dieses wird zufällig generiert und ist somit bei jedem neuen Spielstart anders. Die Spielr starten entsprechend der Teamaufteilung in drei verschiedenen Spawnräumen. Ziel ist die Schatzkammer in der Mitte des Labyrinthes.\\

\section{Rollenverteilung und Charaktere}
In unserem Spiel existiert eine Vielzahl an spielbaren Klassen.\\

\subsection{Der Drache} 
Der Beschützer des Schatzes und der Herr des Labyrinths. Er agiert als Game Master. Seine Strategie ist die Verteidigung. Angreifen kann er durch das Speien von Feuer. Zusätzlich kann er auch Monster spawnen.\\

\subsection{Fernkampf}

\subsubsection{Schamane}
Der Schamane ist ein Magier, sein Angriff ist das Schießen von magischen Sphären.\\

\subsubsection{Bogenschütze}
Der Bogenschütze greift klassisch mit Pfeilen an.\\

\subsection{Mittelfernkampf}

\subsubsection{Ninja}
Der Ninja greift mit dem Werfen von Shuriken an.\\

\subsubsection{Hexe}
Die Hexe ist ein Magier, ihr Angriff ist das Schießen von magischen Sphären.\\

\subsection{Nahkampf}

\subsubsection{Schwertkämpfer}
Der Schwertkämpfer kämpft mit dem Schwert.\\

\subsubsection{Kämpfer}
Der Kämpfer kämpft mit dem nackten Fäusten.\\

\section{Spielende}
Ziel ist das Erbeuten des Schatzes. Dazu müssen zunächst alle drei Schlüsselfragmente ergattert werden. Zu Beginn des Spiels erhält jedes Team ein Fragment. Stirbt der Schlüsselträger eines Teams, wird das Fragment abgeworfen und kann eingesammelt werden. Der getötete Spieler respawnt nach einigen Sekunden wieder am Startpunkt. Ist ein Team im Besitz aller drei Fragmente, muss danach noch die Schatzkammer bertreten werden. Erst danach kann das Spiel beendet werden.\\

\chapter{Design}
Bei dem Spiel handelt es sich um ein dungeonbasiertes Top Down Game im Pixellook. Die Sprites für das Labyrinth sind Open Source, mit einigen eigenen Erweiterungen. Alle Charakter und UI Elemente wurden von uns entworfen.\\

\section{Welt}
	\begin{center}
			\includegraphics[scale=0.8]{maze.png}\\
		Die Schatzkammer\\\

		\includegraphics[scale=0.8]{mazewithui.png}\\
		Spawnroom mit überlappendem Interface\\
	\end{center}

\section{Spielfiguren}
\begin{center}
	\includegraphics[scale=2]{Dragon}\\
	Der Drache\\
		
	\includegraphics[scale=2]{Knight}
	\includegraphics[scale=2]{Fighter}\\
	Der Schwertkämpfer und der Faustkämpfer\\
	
	\includegraphics[scale=2]{Witch}
	\includegraphics[scale=3.5]{Ninja}\\
	Die Hexe und der Ninja\\
	
	\includegraphics[scale=2]{Archer}
	\includegraphics[scale=2]{Shaman}\\
	Der Bogenschütze und der Schamane\\

	\end{center}
\chapter{Technologie und Steuerung}
Das Spiel ist sowohl auf PC, als auch Android spielbar. Die Oberfläche unterschiedet sich dabei nicht. Die Steuerung hingegen ist entsprechend angepasst. Gespielt wird auf dem PC mit Maus und einigen Tasten, auf Android komplett über Touch.\\
Voraussetzung für das Spiel ist ein Endgerät mit respektive Windows 7 oder höher, Linux mit Kernel 3.16 oder höher, Android 4 oder höher. Das Gerät muss Open GL unterstützen und netzwerkfähig sein. Oracle Java muss mindestens in Version 1.6 installiert sein. Für ein gutes Spielerlebnis werden mindestens Geräte mit 2 Prozessorkernen > 1GHz und einer Bildschirmauflösung von 480x800 Bildpunkten empfohlen.

\chapter{Planung und Entwicklung}
\section{ Prototypen}
Im Prototypen entschieden wir uns zunächst für getrennte Anwendungen. So entstanden individuell ein Prototyp für die Bewegung der Spielfigur, die Game Master Fähigkeiten, die Generierung eines simplen zufälligen Labyrinthes und das Netzwerk.\\
Im Anschluss führten wir die einzelnen Prototypen zusammen und arbeiteten an einer gemeinsamen Anwendung weiter.

\section{Alpha}

\section{Beta}

\section{Endprodukt}

\chapter{Entwurfsentscheidungen}

\subsubsection{Das Labyrinth und die Minimap}
Das Labyrinth wird mit Start des Servers generiert und an die Clients übergeben. Grundlage bilden vorgefertigte TiledMaps im Format 32 * 32 Kacheln mit 32 * 32 Pixeln pro Kachel. In der Generierung werden die vorgefertigten Teile zufällig aneinander gesetzt, die Spawnräume und Schatzkammer platziert und anschließend als eine Map gespeichert. Das Labyrinth ist dabei duplikatsfrei.\\
An das Labyrinth geknüpft ist die Minimap, sie wird direkt danach erzeugt und als eine Anordung von Images in Box2D dargestellt. Die Verwaltung von Standorten und aufgedeckten Teilen des Labyrinths übernimmt jeder Client separat.\

\subsubsection{Charakter}
TODO

\subsubsection{Kampfsystem} 
TODO

\subsubsection{Netzwerkkommunikation}
TODO

\subsubsection{Itemsystem}
TODO

\subsubsection{Interface}
TODO

\chapter{Offene Punkte}
\subsubsection{Scissor Stack}
Eine deutliche Performanceoptimierung wäre durch den Einsatz von Scissor Stack möglich. Dies haben wir leider zeitlich nicht mehr geschafft.

\subsubsection{Ausbau des Itemsystems}
Das Itemsystem ist zum jetzigen Zeitpunkt noch sehr minimalistisch. Eine Erweiterung um neue Gegenstände wäre denkbar und einfach zu realisieren.
	
\end{flushleft}
\end{document}
